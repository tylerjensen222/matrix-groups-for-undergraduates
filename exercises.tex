%==============================================================================
%  Matrix Groups for Undergraduates - Exercises
%  Initial LaTeX Commit
%==============================================================================

\documentclass[12pt]{book}

%------------------------------------------------------------------------------
%   Packages
%------------------------------------------------------------------------------
\usepackage[utf8]{inputenc}
\usepackage[T1]{fontenc}
\usepackage{lmodern}

\usepackage{amsmath,amssymb,amsthm}
\usepackage{geometry}
\geometry{margin=1in}

%------------------------------------------------------------------------------
%   Theorem-Like Environments
%------------------------------------------------------------------------------
\theoremstyle{definition}
\newtheorem{exercise}{Exercise}[chapter]

% If you prefer a separate counter per section:
% \newtheorem{exercise}{Exercise}[section]

% Custom solution environment
\newenvironment{solution}
{%
  \par\noindent\textbf{Solution.}\quad
}
{%
  \qed\par
}

%------------------------------------------------------------------------------
%   Metadata
%------------------------------------------------------------------------------
\title{Exercises from \textit{Matrix Groups for Undergraduates} \\
       by Kristopher Tapp}
\author{Tyler Jensen | tyjensen222@gmail.com}
\date{\today}

%------------------------------------------------------------------------------
%   Document
%------------------------------------------------------------------------------
\begin{document}

%--- General notation for fields, sets, and rings you might use frequently:
\newcommand{\R}{\mathbb{R}}    % Real numbers
\newcommand{\C}{\mathbb{C}}    % Complex numbers
\newcommand{\Q}{\mathbb{Q}}    % Rational numbers
\newcommand{\Z}{\mathbb{Z}}    % Integers
\newcommand{\N}{\mathbb{N}}    % Natural numbers

%--- General Linear, Special Linear, etc. 
\newcommand{\GL}[2]{\mathrm{GL}_{#1}\!\bigl(#2\bigr)} % e.g. \GL{n}{\R} -> GL_n(R)
\newcommand{\SL}[2]{\mathrm{SL}_{#1}\!\bigl(#2\bigr)} % e.g. \SL{n}{\C} -> SL_n(C)

%--- Orthogonal, Special Orthogonal, Symplectic
\newcommand{\Ogroup}[1]{\mathrm{O}\!\bigl(#1\bigr)}   % O(n)
\newcommand{\SO}[1]{\mathrm{SO}\!\bigl(#1\bigr)}      % SO(n)
\newcommand{\Sp}[1]{\mathrm{Sp}\!\bigl(#1\bigr)}      % Sp(n) for symplectic group

%--- Unitary, Special Unitary
\newcommand{\U}[1]{\mathrm{U}\!\bigl(#1\bigr)}        % U(n)
\newcommand{\SU}[1]{\mathrm{SU}\!\bigl(#1\bigr)}      % SU(n)

%--- Lie algebras (e.g., \so(n), \sp(n), \su(n), etc.)
\newcommand{\lie}[1]{\mathfrak{#1}}                    % generic Lie algebra symbol
\newcommand{\so}[1]{\mathfrak{so}_{#1}}                % so(n)
\newcommand{\su}[1]{\mathfrak{su}_{#1}}                % su(n)
\newcommand{\gl}[1]{\mathfrak{gl}_{#1}}                % gl(n)
\newcommand{\slalg}[1]{\mathfrak{sl}_{#1}}             % sl(n)
\newcommand{\spalg}[1]{\mathfrak{sp}_{#1}}             % sp(n)

% Optionally define a quick macro for the “big G” or “big g”
\newcommand{\g}{\mathfrak{g}}                          % generic Lie algebra g
\newcommand{\G}{\mathrm{G}}                            % generic Lie group G

\frontmatter
\maketitle
\tableofcontents

\mainmatter

%==============================================================================
\chapter{Matrices}
%==============================================================================
\section{Exercises}

\begin{exercise}
Describe a natural 1-to-1 correspondence between elements of $\text{SO}(3)$ and 
\end{exercise}

\begin{solution}
Placeholder for your solution.
\end{solution}

%==============================================================================
\chapter{All matrix groups are real matrix groups}
%==============================================================================
\section{Exercises}

\begin{exercise}
Another exercise goes here.
\end{exercise}

\begin{solution}
Placeholder for your solution.
\end{solution}

%==============================================================================
\chapter{The orthogonal groups}
%==============================================================================
\section{Exercises}

\begin{exercise}
Another exercise goes here.
\end{exercise}

\begin{solution}
Placeholder for your solution.
\end{solution}

%==============================================================================
\chapter{The topology of matrix groups}
%==============================================================================
\section{Exercises}

\begin{exercise}
Another exercise goes here.
\end{exercise}

\begin{solution}
Placeholder for your solution.
\end{solution}

%==============================================================================
\chapter{Lie algebras}
%==============================================================================
\section{Exercises}

\begin{exercise}
Another exercise goes here.
\end{exercise}

\begin{solution}
Placeholder for your solution.
\end{solution}

%==============================================================================
\chapter{Matrix exponentiation}
%==============================================================================
\section{Exercises}

\begin{exercise}
Another exercise goes here.
\end{exercise}

\begin{solution}
Placeholder for your solution.
\end{solution}

%==============================================================================
\chapter{Matrix groups are manifolds}
%==============================================================================
\section{Exercises}

\begin{exercise}
Another exercise goes here.
\end{exercise}

\begin{solution}
Placeholder for your solution.
\end{solution}

%==============================================================================
\chapter{The Lie bracket}
%==============================================================================
\section{Exercises}

\begin{exercise}
Another exercise goes here.
\end{exercise}

\begin{solution}
Placeholder for your solution.
\end{solution}

%==============================================================================
\chapter{Maximal tori}
%==============================================================================
\section{Exercises}

\begin{exercise}
Another exercise goes here.
\end{exercise}

\begin{solution}
Placeholder for your solution.
\end{solution}

%==============================================================================
\chapter{Homogeneous manifolds}
%==============================================================================
\section{Exercises}

\begin{exercise}
Another exercise goes here.
\end{exercise}

\begin{solution}
Placeholder for your solution.
\end{solution}

%==============================================================================
\chapter{Roots}
%==============================================================================
\section{Exercises}

\begin{exercise}
Another exercise goes here.
\end{exercise}

\begin{solution}
Placeholder for your solution.
\end{solution}

%==============================================================================
\end{document}
