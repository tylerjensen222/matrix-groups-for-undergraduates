%==============================================================================
%  Matrix Groups for Undergraduates - Exercises
%  Initial LaTeX Commit
%==============================================================================

\documentclass[12pt]{book}

%------------------------------------------------------------------------------
%   Packages
%------------------------------------------------------------------------------
\usepackage[utf8]{inputenc}
\usepackage[T1]{fontenc}
\usepackage{lmodern}

\usepackage{amsmath,amssymb,amsthm}
\usepackage{geometry}
\geometry{margin=1in}

%------------------------------------------------------------------------------
%   Theorem-Like Environments
%------------------------------------------------------------------------------
\theoremstyle{definition}
\newtheorem{exercise}{Exercise}[chapter]

% If you prefer a separate counter per section:
% \newtheorem{exercise}{Exercise}[section]

% Custom solution environment
\newenvironment{solution}
{%
  \par\noindent\textbf{Solution.}\quad
}
{%
  \qed\par
}

%------------------------------------------------------------------------------
%   Metadata
%------------------------------------------------------------------------------
\title{Exercises from \textit{Matrix Groups for Undergraduates} \\
       by Kristopher Tapp}
\author{Tyler Jensen | tyjensen222@gmail.com}
\date{\today}

%------------------------------------------------------------------------------
%   Document
%------------------------------------------------------------------------------
\begin{document}

%------------------------------------------------------------------------------
%   Common Macros for Matrix Groups and Lie Algebras
%------------------------------------------------------------------------------
\newcommand{\R}{\mathbb{R}}        % Real numbers
\newcommand{\C}{\mathbb{C}}        % Complex numbers
\newcommand{\Q}{\mathbb{Q}}        % Rational numbers
\newcommand{\Z}{\mathbb{Z}}        % Integers
\newcommand{\N}{\mathbb{N}}        % Natural numbers

%--- General Linear, Special Linear, etc. 
\newcommand{\GL}[2]{\mathrm{GL}_{#1}(#2)}   % e.g. \GL{n}{\R} -> GL_n(\R)
\newcommand{\SL}[2]{\mathrm{SL}_{#1}(#2)}   % e.g. \SL{n}{\C} -> SL_n(\C)

%--- Orthogonal, Special Orthogonal, Symplectic
\newcommand{\Ogroup}[1]{\mathrm{O}(#1)}     % O(n)
\newcommand{\SO}[1]{\mathrm{SO}(#1)}        % SO(n)
\newcommand{\Sp}[1]{\mathrm{Sp}(#1)}        % Sp(n) symplectic group

%--- Unitary, Special Unitary
\newcommand{\U}[1]{\mathrm{U}(#1)}          % U(n)
\newcommand{\SU}[1]{\mathrm{SU}(#1)}        % SU(n)

%--- Lie algebras
\newcommand{\lie}[1]{\mathfrak{#1}}         % \lie{g} -> \mathfrak{g}
\newcommand{\so}[1]{\mathfrak{so}_{#1}}     % so(n)
\newcommand{\sualg}[1]{\mathfrak{su}_{#1}}  % su(n)
\newcommand{\glalg}[1]{\mathfrak{gl}_{#1}}  % gl(n)
\newcommand{\slalg}[1]{\mathfrak{sl}_{#1}}  % sl(n)
\newcommand{\spalg}[1]{\mathfrak{sp}_{#1}}  % sp(n)

%--- Generic placeholders for a Lie algebra g and Lie group G
\newcommand{\g}{\mathfrak{g}}
\newcommand{\G}{\mathrm{G}}


\frontmatter
\maketitle
\tableofcontents

\mainmatter

%==============================================================================
\chapter{Matrices}
%==============================================================================

% Exercise 1.1
\begin{exercise}
Describe a natural 1-to-1 correspondence between elements of $\SO{3}$ and elements of

\[
T^1S^2 = \{ (p, v) \in \R^3 \times \R^3 : |p| = |v| = 1 \text{ and } p \perp q\}
\]

\end{exercise}

\begin{solution}
Using the globe analogy from Question 1.2, fix a point $r$ to be the north pole, 
and a point $e$ that lies on the equator induced by the choice of $r$, and assert this as the arbitrary `identity'.

Next, given some $A \in \SO{3}$, identify an element in $T^1S^2$ via $A \mapsto (Ar, Av)$, 
as in first where $A$ maps the north pole $r$, and then how $A$ rotates the globe about the axis induced by $r$ and its antipodal point.
\end{solution}

%==============================================================================
\chapter{All matrix groups are real matrix groups}
%==============================================================================
\section{Exercises}

\begin{exercise}
Another exercise goes here.
\end{exercise}

\begin{solution}
Placeholder for your solution.
\end{solution}

%==============================================================================
\chapter{The orthogonal groups}
%==============================================================================
\section{Exercises}

\begin{exercise}
Another exercise goes here.
\end{exercise}

\begin{solution}
Placeholder for your solution.
\end{solution}

%==============================================================================
\chapter{The topology of matrix groups}
%==============================================================================
\section{Exercises}

\begin{exercise}
Another exercise goes here.
\end{exercise}

\begin{solution}
Placeholder for your solution.
\end{solution}

%==============================================================================
\chapter{Lie algebras}
%==============================================================================
\section{Exercises}

\begin{exercise}
Another exercise goes here.
\end{exercise}

\begin{solution}
Placeholder for your solution.
\end{solution}

%==============================================================================
\chapter{Matrix exponentiation}
%==============================================================================
\section{Exercises}

\begin{exercise}
Another exercise goes here.
\end{exercise}

\begin{solution}
Placeholder for your solution.
\end{solution}

%==============================================================================
\chapter{Matrix groups are manifolds}
%==============================================================================
\section{Exercises}

\begin{exercise}
Another exercise goes here.
\end{exercise}

\begin{solution}
Placeholder for your solution.
\end{solution}

%==============================================================================
\chapter{The Lie bracket}
%==============================================================================
\section{Exercises}

\begin{exercise}
Another exercise goes here.
\end{exercise}

\begin{solution}
Placeholder for your solution.
\end{solution}

%==============================================================================
\chapter{Maximal tori}
%==============================================================================
\section{Exercises}

\begin{exercise}
Another exercise goes here.
\end{exercise}

\begin{solution}
Placeholder for your solution.
\end{solution}

%==============================================================================
\chapter{Homogeneous manifolds}
%==============================================================================
\section{Exercises}

\begin{exercise}
Another exercise goes here.
\end{exercise}

\begin{solution}
Placeholder for your solution.
\end{solution}

%==============================================================================
\chapter{Roots}
%==============================================================================
\section{Exercises}

\begin{exercise}
Another exercise goes here.
\end{exercise}

\begin{solution}
Placeholder for your solution.
\end{solution}

%==============================================================================
\end{document}
